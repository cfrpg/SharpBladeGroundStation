\section{需求}
\subsection{所需的状态的方式}
暂时无法理解“状态”的含义

\subsection{功能需求}
\subsubsection{飞控连接与通信功能}
地面站系统使用串口通过直接连接或经由数传电台与无人机飞控进行通信。
地面站软件只会同时与一台无人机建立连接。

当连接正常接入时,软件能够自动识别连接使用的串口,接入的硬件和使用的通讯协议,并在10s内与飞控建立连接。
连接建立后忽略所有新出现的串口连接。软件通过检测心跳包判断连接是否可用,当一定时间没有接收到心跳包后认为连接断开。心跳包丢失时间可在软件中进行设置。

在建立连接前,若存在无法识别通信协议的串口连接,软件将持续对串口进行扫描,直到出现可识别的串口。默认串口通信波特率为115200,若使用其他波特率进行通信,软件提供手动连接界面以设置自定义串口连接。

\subsubsection{飞行状态参数获取,显示与记录}
地面站系统建立连接后与无人机飞控进行通信,解析数据包获取飞行数据,并以一定的方式将其中的部分参数显示在软件界面中。界面中直接显示的数据见表\ref{t3dppara}。
对于接收到的数据,软件会以与飞控机载日志相同的格式进行记录,并将记录保存在硬盘中。
所有数值全部使用SI单位。

对于不同的飞控硬件,发回的数据内容可能并不相同。若飞控没有发送需要显示的参数,地面站中将显示参数的缺省值。
若飞控中途断开连接,软件将保持显示最后一次接收到的数值。

\begin{table}[ht]
\centering
\caption{显示的飞行参数}
\label{t3dppara}
\begin{tabular}{|l|l|}
\hline
\multicolumn{1}{|c|}{参数类型} & \multicolumn{1}{c|}{参数名}  \\ \hline
飞行姿态 & 俯仰角,滚转角,机头指向              \\ \hline
飞行状态 & 空速,地速,航向,上升/下降率,相对高度,飞行模式 \\ \hline
定位状态 & 精度,纬度,卫星数,定位状态,定位误差       \\ \hline
导航状态 & 航程,航时,回航角,航点信息            \\ \hline
动力状态 & 电池电压,电池电流,估计剩余电量          \\ \hline
通信状态 & 通信速率,通信协议,RSSI            \\ \hline
\end{tabular}
\end{table}

\subsubsection{地图显示}
地面站系统能够显示普通地图,卫星地图或混合地图,并能够在三种地图之间切换。地图显示部分占据地面站界面的主要部分,能够自由进行移动和缩放。
除地图外,还能够在地图上实时显示无人机位置,方向,飞行轨迹和预设航线。

\subsubsection{}

\subsection{外部接口需求}

\subsubsection{接口标识和接口图}

\subsubsection{MAVLINK 1.0}


\subsubsection{接口2}

\subsection{CSCI内部接口需求}

\subsection{CSCI内部数据需求}

\subsection{适应性需求}

\subsection{安全性需求}

\subsection{保密性需求}
无此部分

\subsection{CSCI环境需求}

\subsection{计算机资源需求}
\subsubsection{计算机硬件需求}

\subsubsection{计算机硬件资源使用需求}

\subsubsection{计算机软件需求}

\subsubsection{计算机通信需求}

\subsection{软件质量因素}

\subsection{设计和实现约束}

\subsection{人员需求}
大概也无此部分

\subsection{其他需求}



\endinput