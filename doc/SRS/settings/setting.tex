%---------------------------------纸张大小设置---------------------------------%
\usepackage{geometry}
    \geometry{left=1in,right=1in,top=1in,bottom=2in}
%------------------------------------------------------------------------------%


%----------------------------------必要库支持----------------------------------%
\usepackage{amsmath}
\usepackage{amssymb}
\usepackage{amsfonts}
\usepackage{bm}
\usepackage{xcolor}
\usepackage{tikz}
\usepackage{layouts}
\usepackage[numbers,sort&compress]{natbib}
\usepackage{clrscode}
\usepackage{gensymb}
\usepackage[final]{pdfpages}
%------------------------------------------------------------------------------%


%--------------------------------设置标题与目录--------------------------------%
\usepackage[sf]{titlesec}
\usepackage{titletoc}
%------------------------------------------------------------------------------%


%--------------------------------添加书签超链接--------------------------------%
\usepackage[unicode=true,colorlinks=false,pdfborder={0 0 0}]{hyperref}
    % 在此处修改打开文件操作
    \hypersetup{
        bookmarks=true,         % show bookmarks bar?
        pdftoolbar=true,        % show Acrobat’s toolbar?
        pdfmenubar=true,        % show Acrobat’s menu?
        pdffitwindow=true,      % window fit to page when opened
        pdfstartview={FitH},    % fits the width of the page to the window
        pdfnewwindow=true,      % links in new PDF window
    }
    % 在此处添加文章基础信息
    \hypersetup{
        pdftitle={title},
        pdfauthor={author},
        pdfsubject={subject},
        pdfcreator={creator},
        pdfproducer={producer},
        pdfkeywords={key1  key2  key3}
    }
%------------------------------------------------------------------------------%


%---------------------------------设置字体大小---------------------------------%
\usepackage{type1cm}
\newcommand{\sChuhao}{\fontsize{42pt}{63pt}\selectfont}         % 初号, 1.5倍
\newcommand{\sYihao}{\fontsize{25pt}{36pt}\selectfont}          % 一号, 1.4倍
\newcommand{\sErhao}{\fontsize{22pt}{28pt}\selectfont}          % 二号, 1.25倍
\newcommand{\sXiaoer}{\fontsize{18pt}{18pt}\selectfont}         % 小二, 单倍
\newcommand{\sSanhao}{\fontsize{16pt}{24pt}\selectfont}         % 三号, 1.5倍
\newcommand{\sXiaosan}{\fontsize{15pt}{22pt}\selectfont}        % 小三, 1.5倍
\newcommand{\sSihao}{\fontsize{14pt}{21pt}\selectfont}          % 四号, 1.5倍
\newcommand{\sBSihao}{\fontsize{14pt}{17.5pt}\selectfont}		% 四号, 1.25倍
\newcommand{\sHalfXiaosi}{\fontsize{12.5pt}{16.25pt}\selectfont}% 半小四, 1倍
\newcommand{\sBXiaosi}{\fontsize{13pt}{19pt}\selectfont}   		% 半小四, 1.5倍
\newcommand{\sXiaosi}{\fontsize{12pt}{14.4pt}\selectfont}       % 小四, 1.25倍
\newcommand{\sDXiaosi}{\fontsize{12pt}{16pt}\selectfont}        % 小四, 单倍
\newcommand{\sLargeWuhao}{\fontsize{11pt}{11pt}\selectfont}     % 大五, 单倍
\newcommand{\sWuhao}{\fontsize{10.5pt}{10.5pt}\selectfont}      % 五号, 单倍
\newcommand{\sXiaowu}{\fontsize{9pt}{9pt}\selectfont}           % 小五, 单倍
%------------------------------------------------------------------------------%


%---------------------------------设置中文字体---------------------------------%
\usepackage{fontspec}
\usepackage[SlantFont,BoldFont,CJKchecksingle,CJKnumber]{xeCJK}
% 使用 Adobe 字体
\newcommand\adobeSog{Adobe Song Std}
\newcommand\adobeHei{Adobe Heiti Std}
\newcommand\adobeKai{Adobe Kaiti Std}
\newcommand\adobeFag{Adobe Fangsong Std}
\newcommand\codeFont{Consolas}
% 设置字体
\defaultfontfeatures{Mapping=tex-text}
\setCJKmainfont[ItalicFont=\adobeKai, BoldFont=\adobeHei]{\adobeSog}
\setCJKsansfont[ItalicFont=\adobeKai, BoldFont=\adobeHei]{\adobeSog}
\setCJKmonofont{\codeFont}
\setmonofont{\codeFont}
% 设置字体族
\setCJKfamilyfont{song}{\adobeSog}      % 宋体  
\setCJKfamilyfont{hei}{\adobeHei}       % 黑体  
\setCJKfamilyfont{kai}{\adobeKai}       % 楷体  
\setCJKfamilyfont{fang}{\adobeFag}      % 仿宋体
\setCJKfamilyfont{nwpu}{nwpuname}
% 新建字体命令,统一前缀f(a.k.a font)
\newcommand{\fSong}{\CJKfamily{song}}
\newcommand{\fHei}{\CJKfamily{hei}}
\newcommand{\fFang}{\CJKfamily{fang}}
\newcommand{\fKai}{\CJKfamily{kai}}
\newcommand{\fNWPU}{\CJKfamily{nwpu}}
%------------------------------------------------------------------------------%


%------------------------------添加插图与表格控制------------------------------%
\usepackage{graphicx}
\usepackage[font=small,labelsep=quad]{caption}
\usepackage{wrapfig}
\usepackage{multirow,makecell}
\usepackage{longtable}
\usepackage{booktabs}
\usepackage{tabularx}
\usepackage{setspace}
%------------------------------------------------------------------------------%


%---------------------------------添加列表控制---------------------------------%
\usepackage{enumerate}
\usepackage{enumitem}
%------------------------------------------------------------------------------%


%---------------------------------设置引用格式---------------------------------%
\renewcommand\figureautorefname{图}
\renewcommand\tableautorefname{表}
\renewcommand\equationautorefname{式}
\numberwithin{equation}{section}
\numberwithin{table}{section}
\numberwithin{figure}{section}
\newcommand\myreference[1]{[\ref{#1}]}
\newcommand\eqrefe[1]{式(\ref{#1})}
\renewcommand\theequation{\thesection-\arabic{equation}}
% 增加 \ucite 命令使显示的引用为上标形式
\newcommand{\ucite}[1]{$^{\mbox{\scriptsize \cite{#1}}}$}
%------------------------------------------------------------------------------%


%--------------------------------设置定理类环境--------------------------------%
\usepackage[amsmath,thmmarks]{ntheorem}
\newtheorem{myexample}{例}
\newtheorem{thm}{定理}[section]
\newtheorem{defi}{定义}[section]
%------------------------------------------------------------------------------%


%--------------------------设置中文段落缩进与正文版式--------------------------%
\XeTeXlinebreaklocale "zh"       %使用中文的换行风格
\XeTeXlinebreakskip = 0pt plus 1pt    %调整换行逻辑的弹性大小
%\xeCJKcaption{gb_452}
\usepackage{indentfirst}
\setlength{\parindent}{26pt}
%\renewcommand\chaptername{\CJKprechaptername\CJKthechapter\CJKchaptername}
\setlength{\parskip}{3pt plus1pt minus1pt} % 段落间距
\renewcommand{\baselinestretch}{1.25} % 行距
%------------------------------------------------------------------------------%


%----------------------------设置段落标题与目录格式----------------------------%
\setcounter{secnumdepth}{3}
\setcounter{tocdepth}{3}
\usepackage{CJKnumb}
%\renewcommand{\chaptername}{第\CJKnumber{\thechapter}章}
\renewcommand{\figurename}{图}
\renewcommand{\tablename}{表}
\renewcommand{\bibname}{参考文献}
\renewcommand\contentsname{\hspace*{\fill}目\quad 录\hspace*{\fill}}
\newcommand{\keywords}[1]{\\ \\ \textbf{关~键~词}:#1}

%\titleformat{\chapter}[hang]{\normalfont\sSanhao\filcenter\fHei\bf}{\sSanhao{\chaptertitlename}}{20pt}{\sSanhao}
%\renewcommand{\thesubsection}{\arabic{subsection}.}
%\titleformat{\section}[hang]{\fHei \bf \sSanhao\filcenter}{\sXiaosan \CJKnumber{\thesection}、}{0em}{}{}
\titleformat{\section}[hang]{\fHei \bf \sSanhao}
    {\sSanhao \thesection }{0.5em}{}{}
\titleformat{\subsection}[hang]{\fHei \bf \sXiaosan}
    {\sXiaosan \thesubsection}{0.5em}{}{}
\titleformat{\subsubsection}[hang]{\fHei \bf}
    {\sXiaosan \thesubsubsection}{0.5em}{}{}
% 目录格式
%\titlespacing{\chapter}{0pt}{-2ex  plus .1ex minus .2ex}{0.25em}
%\titlespacing{\chapter}{0pt}{1ex}{0.5em}
\titlespacing{\section}{0pt}{0.5em}{0.5em}
\titlespacing{\subsection}{0pt}{0.2em}{0em}
\titlespacing{\subsubsection}{0pt}{0.25em}{0pt}
% 缩小目录中各级标题之间的缩进
%\dottedcontents{chapter}[0.0em]{\fHei\vspace{0.5em}}{0.0em}{5pt}
\dottedcontents{section}[1.16cm]{}{1.8em}{5pt}
\dottedcontents{subsection}[2.00cm]{}{2.7em}{5pt}
\dottedcontents{subsubsection}[2.86cm]{}{3.4em}{5pt}
%------------------------------------------------------------------------------%

%---------------------------------设置页眉页脚---------------------------------%
\usepackage{fancyhdr}
\usepackage{fancyref}
%\addtolength{\headsep}{-0.1cm}          %页眉位置
%\addtolength{\footskip}{-0.1cm}         %页脚位置
\addtolength{\topmargin}{0.5cm}
\newcommand{\makeheadrule}{
    \makebox[0pt][l]{\rule[.7\baselineskip]{\headwidth}{0.8pt}}
    \vskip-.8\baselineskip
}
\makeatletter
\renewcommand{\headrule}{%
    {
        \if@fancyplain\let\headrulewidth\plainheadrulewidth\fi
        \makeheadrule
    }
}
%\pagestyle{fancyplain}
\pagestyle{plain}
\fancyhf{}
\fancyfoot[C,C]{\sWuhao-~\thepage~-}


%------------------------------------------------------------------------------%




%------------------------------------------------------------------------------%

%----------------------------------其他补充设置--------------------------------%

% 下划线
\newcommand\dlmu@underline[2][5cm]{\hskip1pt\underline{\hb@xt@ #1{\hss#2\hss}}\hskip3pt}
\let\coverunderline\dlmu@underline

\allowdisplaybreaks[4]
%------------------------------------------------------------------------------%



%----------------------------------添加代码控制--------------------------------%
\usepackage{listings}
\lstset{
    basicstyle=\footnotesize\ttfamily,
    numbers=left,
    numberstyle=\tiny,
    numbersep=5pt,
	framexleftmargin=10mm,
	frame=none,
    tabsize=4,
    extendedchars=true,
    breaklines=true,
    keywordstyle=\color{blue},
    numberstyle=\color{purple},
    commentstyle=\color[RGB]{0,96,96},
    stringstyle=\color{orange}\ttfamily,
    showspaces=false,
    showtabs=false,
    framexrightmargin=5pt,
    framexbottommargin=4pt,
    showstringspaces=false
    escapeinside=`', %逃逸字符(1左面的键),用于显示中文
}
\renewcommand{\lstlistingname}{CODE}
\lstloadlanguages{MATLAB}
%------------------------------------------------------------------------------%

\renewcommand\arraystretch{1.2}
\renewcommand{\thefigure}{\thesection-\arabic{figure}}
\renewcommand{\thetable}{\thesection-\arabic{table}}
\captionsetup[table]{labelfont=bf,textfont=bf}
\graphicspath{{figures/}}

\newcommand{\nl}{\vspace{3ex}}


\endinput